\chapter{Bitcoin}\label{ch:bitcoin}

Operating systems make it easier to develop software.
The easier it is to develop software, the more software we develop, which makes computers more useful to humans.
As a result, human society is transforming rapidly.
Two decades ago, we had to carefully plan in advance how we meet; now, we do it by mobile phone, as we go.
One decade ago, we learned what happens in the world by reading the news, which were reported by professional journalists; now, we consume social media, which is produced by ourselves.
One decade ago, we planned our holidays using travel guides and paper maps; now, we use online reviews and Google maps.
Being always connected, to your friends and to your hive, really makes a difference in how we live our lives.
What is the next big change, the next killer app?

Alas, I do not have a crystal ball.
But, I can make a few observations about the changes that did occur in the past couple of decades.
First, we can see that change happens on the timescale of about a decade.
This is quite fast if put in historical perspective: human societies used to go on for centuries before a significant change could be observed.
But, it is still slow enough that we cannot pinpoint the one moment when a change occurs.
For example, Twitter appeared in 2006.
Not many people thought that Twitter will change how a billion of people go about their lives.
Obviously, a few people did think so, but they mostly fell in two categories: those with a financial interest in Twitter's success, and those who are so enthusiastic that they think any new gadget is cool and will change the world.
Nowadays, it is undeniable that Twitter has a huge influence on daily lives.
When did Twitter go from `cute new idea that maybe will work' to `the oracle that hundreds of million people religiously consult every day'?
Hard to tell; but, sometime between 2006 and 2017.

Second, the change is usually not about a completely new idea.
Rather, it is about doing something that humans have been doing for a long time but doing it in a different way.
To take Twitter as an example again, what people do on Twitter is essentially chatting.
People have used chatting -- idle conversation about whatever crosses your mind -- for millennia.
The effect of chatting still is what it always was: sometimes it bring communities together around common ideals, other times it causes conflicts.
But, on Twitter, chatting is taken to a new level: it's not family chatting or village chatting, it's global chatting.
This quantitative difference makes a qualitative difference: chatting on Twitter \emph{feels} different.

In summary, we are looking for some widespread human habit that might change because of computers, on a timescale of $\sim10$~years.
Here's two candidates: cars and money.
So far, computers did not change in a fundamental way how we use cars; but, they threaten to replace drivers.
So far, computers did change how we use money: online commerce and credit/debit cards made us use physical money a lot less often.
But, digital money like bitcoin and ethereum threaten to bring more fundamental changes to how we use money.

This chapter will discuss how digital money works.
Before getting there, let us briefly reflect on how traditional money works.
Money is a human invention.
Initially, it solved a pragmatic problem: it reduced the complexity of the trading problem.
If Alice has onions and Bob has tomatoes, but Alice wants tomatoes and Bob wants onions, then then can meet and trade goods.
If Alice and Bob are the only characters things are simple, but then Courtney comes along the next day with potatoes.
Today, Alice still wants tomatoes, but Bob wants potatoes, and Courtney would rather have onions.
This is the OTP trading problem ({\bf o}nions, {\bf t}omatoes, {\bf p}otatoes).
To solve it, Alice, Bob and Courtney need to all get together and exchange goods.
Of course, to do this, they might first communicate their needs to each-other.
If only money had been invented!
In that case, Alice, Bob, and Courtney don't even need to know each-other.
They each go to the market and sell (for money) their surplus of vegetables.
Then, when a need arises, they go back to the market and buy (with money) what they need.
In a sense, the situation is more complicated: we need to track this thing called money, potatoes don't go directly from Courtney to Bob (but use the market as an intermediary), and the market will keep some of the vegetables in exchange for the service it provides.
But, from the point of view of each of Alice, Bob, and Courtney, the situation is vastly simpler: they just need to interact with the market, using money.
In particular, trade is just as easy in a village with 100~people as it is in a city with $10{,}000{,}000$ people.
This is what reducing trade complexity means.

Why does money work?
It has two important properties: (1)~everybody values money, and (2)~everybody trusts that everybody values money.
Why would I give my tomatoes for a piece of paper with the text $\pounds10$ written on it?
Because I trust that, in the future, other people will want that piece of paper and, in exchange, they will give me something.
If a majority of people will suddenly decide that pounds are worthless, then they \emph{will} become worthless.
We will see that digital money do not solve this problem.

The next step in the evolution of money was the apparition of banks and, more generally, of capitalism.


\section{Cryptographic Primitives}

\section{Coins}

\section{Ledger}

\section{Attacks}

\section{Exercises}


\section{Notes}

For more historical information on money, I recommend Chapters 10~and~16 of \citet{sapiens}, and the YouTube videos that accompany the book.

% vim:spell:spelllang=en_gb:

