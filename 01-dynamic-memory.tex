\chapter{Dynamic Memory Allocation}\label{ch:01-dynamic-memory}

\tikzset{us/.style={fill=red!50}}
\tikzset{fs/.style={fill=green!50}}

\newcommand\mem[1]{
  \foreach \x in {0,1,...,#1} {
    \draw (\x,0) rectangle (\x+1,1);
  }
}
\newcommand\memnum[1]{
  \foreach \x in {0,1,...,#1} {
    \draw[fill=gray] (\x,0) rectangle node{$\x$} (\x+1,1);
  }
}
\newcommand\memblock[3]{
  \draw[#1,very thick,rounded corners=1pt]
    (#2+0.2,0.2) rectangle (#3-0.2,0.8);
}
\newcommand\use[2]{\memblock{us}{#1}{#2}}
\newcommand\fre[2]{\memblock{fs}{#1}{#2}}
\newcommand\T{\tikz[baseline=2,scale=.4]}

\section{High Level Interface: \.{malloc} and \.{free}}

The functions \.{malloc} and \.{free} are typically implemented in user space;
  for example, in the GNU implementation of the standard C library \.{glibc}.
Nevertheless, we will study them because
  (1)~they are part of POSIX,
  (2)~they are simple, and
  (3)~similar ideas are used in later lectures.
If your preferred language is Java,
  then think of \.{x=(T*)malloc(sizeof(T) * n)} as C's counterpart of \.{x=new T[n]}:
  that's how we allocate memory for $n$~objects of type~$T$.
However,
  nothing like \.{free(x)} exists in Java.

As an example, consider the following task.
We are given a file with $n+2$ integers separated by whitespace:
\[
  n\ x_0\ x_1\ \ldots\ x_{n-1}\ j
\]
The first integer is~$n$;
  the next $n$~integers are $x_0,\ldots,x_{n-1}$;
  the last integer is~$j$.
Assume that $0 \le j < n$.
Our program should print~$x_j$.
The problem is that we do not know~$n$ in advance,
  so we do not know how much memory is needed to store all integers.
To solve this problem, we can use \.{malloc}:
\begin{ccode}
#include <stdio.h>
#include <stdlib.h>

int main() {
  int n; scanf("%d", &n);
  int * x = malloc(sizeof(int) * n);
  for (int i = 0; i < n; ++i) scanf("%d",&x[i]);
  int j; scanf("%d", &j);
  printf("%d\n", x[j]);
  free(x);
}
\end{ccode}
The important lines are 6~and~10.
On line~6 we request enough space to store $n$~integers;
  on line~10 we release that space.
Note that the value of $n$ is unknown when we write the program:
  we have to ask the user (on line~5) what the value of~$n$ is.
In general,
  dynamic memory allocation is used when we know only at runtime how much memory is needed.

To use dynamically allocated memory, you only need to understand two functions:
  \.{malloc} and \.{free}.
A call \.{malloc(n)} allocates $n$~bytes of memory,
  and returns a pointer~$x$.
The $n$~allocated bytes are at addresses $x, x+1, x+2,\ldots, x+n-1$.
A call \.{free(x)} deallocates the block of memory that starts at address~$x$.

\smallskip\noindent
\emph{C quirk${}^*$:}\enspace
Pointer arithmetic is slightly different from address arithmetic.
Addresses behave like normal integers.
Variables of type \.{void*} (pointer to void) also behave like normal integers.
But, variables of type \.{T*} (pointer to some non-void type \.{T})
  have a funny addition:
The C expression \.{(int)(x + 42)} equals \.{(int)x + 42 * sizeof(T)}.
The first expression says `add $42$ to the pointer $x$,
  and then convert the result to an integer'.
The second expression says `convert the pointer $x$ to an integer,
  and then add to it $42\times S$,
  where $S$ is the number of bytes necessary to store an object of type~$T$'.
Why is this?
Because we want $x+1$ to point to the object of type~$T$
  that sits next to the object of type~$T$ to which $x$~points,
  without overlap.


\section{Policies: First Fit, Best Fit, Buddy System}

The high level \.{malloc}\slash\.{free} interface
  must be implemented in terms of a lower level interface.
This lower level interface is an array of bytes.
Moreover, we are not allowed to use any other storage space but this array.
In this section, we tackle a simplified problem.
We will just worry about which portions of the array we want to allocate,
  without worrying about which storage space we use for our bookkeeping.
We'll worry about bookkeeping in the next section.

The first policy to know is \emph{first fit}:
  always allocate a block of memory at the smallest address possible.
Here is an example:
\begin{center}\small
\begin{tabular}{@{}lllrrr@{}}
\toprule
Call & Return & Memory  & Free & Available & Fragmentation \\
\midrule
  & & \T{\memnum9} \\
  & & \T{\mem9} & $10$ & $10$ & $0.00$\\
$\.{malloc}(3)$ & $0$ & \T{\mem9\use03} & $7$ & $7$ & $0.00$ \\
$\.{malloc}(2)$ & $3$ & \T{\mem9\use03\use35} & $5$ & $5$ & $0.00$ \\
$\.{malloc}(1)$ & $5$ & \T{\mem9\use03\use35\use56} & $4$ & $4$ & $0.00$\\
$\.{free}(3)$ &  & \T{\mem9\use03\use56} & $6$ & $4$ & $0.33$ \\
$\.{malloc}(3)$ & $6$ & \T{\mem9\use03\use56\use69} & $3$ & $2$ & $0.33$\\
$\.{malloc}(1)$ & $3$ & \T{\mem9\use03\use34\use56\use69} & $2$ & $1$ & $0.50$\\
$\.{free}(3)$ & & \T{\mem9\use03\use56\use69} & $3$ & $2$ & $0.33$\\
$\.{free}(5)$ & & \T{\mem9\use03\use69} & $4$ & $3$ & $0.25$\\
$\.{free}(0)$ & & \T{\mem9\use69} & $7$ & $6$ & $0.17$\\
\bottomrule
\end{tabular}
\end{center}
The call $\.{free}(x)$ refers to the block at address~$x$,
  which was allocated by the last \.{malloc} that returned~$x$.

The next policy to know is next fit,
  but we skip it because it doesn't work that well in practice.
After that comes \emph{best fit}:
  put a block in the smallest gap where it fits.
If there are several smallest gaps, then the one at the smallest address is chosen.
In the example from above,
  best fit would only differ in how it handles the last \.{malloc}:
\begin{center}\small
\begin{tabular}{@{}lllrrr@{}}
\toprule
Call & Return & Memory & Free & Available & Fragmentation \\
\midrule
  & & \T{\memnum9} \\
   &  & \T{\mem9\use03\use56\use69} & $3$ & $2$ & $0.33$\\
$\.{malloc}(1)$ & $9$ & \T{\mem9\use03\use56\use69\use9{10}} & $2$ & $2$ & $0.00$\\
\bottomrule
\end{tabular}
\end{center}

First fit, next fit, best fit and their cousins suffer
  from \emph{external fragmentation}:
  The largest block that can be allocated is smaller
    than the total amount of free memory.
\[
  \text{external fragmentation} =
  1 - \frac{\text{biggest available block}}{\text{total free memory}}
\]

The next policy, (binary) \emph{buddy system},
  tends to have much lower external fragmentation.
The buddy system works when the size of the memory is a power of two,
  and it allocates only blocks whose size is a power of two and are aligned.
For example, if the memory has size $2^4$,
  then \emph{only} the following blocks can be allocated:
\begin{center}\small
\begin{tabular}{@{}lr@{}}
\toprule
Memory & Order \\
\midrule
  \T{\memnum{15}} \\
  \T{\mem{15}\foreach \x in{0,1,...,15}{\fre{\x}{\x+1}}} & $0$ \\
  \T{\mem{15}\foreach \x in{0,2,4,...,14}{\fre{\x}{\x+2}}} & $1$ \\
  \T{\mem{15}\foreach \x in{0,4,8,...,12}{\fre{\x}{\x+4}}} & $2$ \\
  \T{\mem{15}\fre08\fre{8}{16}} & $3$ \\
  \T{\mem{15}\fre{0}{16}} & $4$ \\
\bottomrule
\end{tabular}
\end{center}
Here is how the buddy system handles the use case from before:
\begin{center}\small
\begin{tabular}{@{}lllrrr@{}}
\toprule
 &&& \multicolumn{2}{c}{Fragmentation} \\ \cmidrule(l){4-5}
Call & Return & Memory & Internal & External \\
\midrule
  & & \T{\memnum{15}} \\
  & & \T{\mem{15}\fre{0}{16}} & $0.00$ & $0.00$\\
$\.{malloc}(3)$ & $0$ & \T{\mem{15}\use04\fre48\fre{8}{16}} & $0.33$ & $0.33$ \\
$\.{malloc}(2)$ & $4$ & \T{\mem{15}\use04\use46\fre68\fre{8}{16}} & $0.20$ & $0.20$\\
$\.{malloc}(1)$ & $6$ & \T{\mem{15}\use04\use46\use67\fre78\fre{8}{16}} & $0.17$ & $0.11$\\
$\.{free}(4)$ &  & \T{\mem{15}\use04\fre46\use67\fre78\fre{8}{16}} & $0.25$ & $0.27$\\
$\.{malloc}(3)$ & $8$ & \T{\mem{15}\use04\fre46\use67\fre78\use{8}{12}\fre{12}{16}} & $0.29$ & $0.43$\\
$\.{malloc}(1)$ & $7$ & \T{\mem{15}\use04\fre46\use67\use78\use{8}{12}\fre{12}{16}} & $0.25$ & $0.33$ \\
$\.{free}(7)$ & & \T{\mem{15}\use04\fre46\use67\fre78\use{8}{12}\fre{12}{16}} & $0.29$ & $0.43$\\
$\.{free}(6)$ & & \T{\mem{15}\use04\fre48\use{8}{12}\fre{12}{16}} & $0.33$ & $0.50$\\
$\.{free}(0)$ & & \T{\mem{15}\fre08\use{8}{12}\fre{12}{16}} & $0.33$ & $0.33$\\
\bottomrule
\end{tabular}
\end{center}
It turns out that in this particular example
  the external fragmentation is not smaller than for first fit.
In addition, we also have \emph{internal fragmentation}:
  there is more allocated memory than was requested.
For example, if $3$~cells are requested, then $4$~are allocated,
  because the size of a block must be a power of two.
In the table above, internal fragmentation is computed as follows:
\[
  \text{internal fragmentation} =
  \frac{\text{allocated memory}}{\text{requested memory}} - 1
\]

When $\.{free}(7)$~is called, two gluing operations take place:
  first, the block $[6]$ is glued with its buddy $[7]$ to form the block $[6,7]$;
  second, the block $[4,5]$ is glued with its buddy $[6,7]$
    to form the block $[4,5,6,7]$.
In general, each block of size $2^{k+1}$ that starts at address $n\cdot 2^{k+1}$
  is made out of two \emph{buddies} of size~$2^k$,
  one starting at address $2n\cdot2^k$
  and one starting at address $(2n+1)\cdot2^k$.
Blocks are split into buddies when smaller blocks are needed.
There are never two free buddies --- they get glued into a bigger block.

\medskip
Observe that the policies from above work for arrays of~$x$ for all~$x$,
  not only for $x=\text{bytes}$.
When the type~$x$ is irrelevant, we talk about an `array of cells',
  as opposed to `array of bytes' or `array of integers'.

\section{Implementation}

To implement the policies from the previous section,
  we have some bookkeeping to do.
How do we know which cells are allocated and which are free?
When we are given the start address of a block,
  how do we know how many cells are we supposed to free?
How do we find the free cells?

Before reading on, you may want to try coming up with a solution yourself.
You have to keep in mind that the only storage space you have is the given array.
This means that it's OK to ignore the need for a constant number of variables,
  but not OK to ignore the need for a variable amount of space.
Why?
Well, if you need, say, $10$~integers,
  then you just reserve enough space at the end of the given array,
  and then pretend the array is smaller.
However if the extra space is not a fixed number like~$10$,
  then you need to think carefully about where you put that information.
In particular,
  you can't assume you have something like \.{malloc} (or \.{new}),
  because you don't.

The data structure used most often to solve this problem is a free list.
\todo{def segregated lists}
We will take a look at its structure, and then we'll revisit the policies we saw.
The presentation makes the simplifying assumption
  that each array cell holds an address.
(On the one hand, this is not quite true:
  memory looks more like an array of bytes,
  and one needs $\sim8$~bytes to store an address.
On the other hand,
  removing this simplifying assumption doesn't involve any interesting idea.)

\subsection{Free List}

A free list is just what the name says:
  a list of free blocks of memory.
For example:
\begin{center}
\begin{tikzpicture}[xscale=0.7]
\foreach \addr/\data/\s in {
  0/2/us,
  1/u/us,
  2/3/us,
  3/u/us,
  4/u/us,
  5/4/fs,
  6/11/fs,
  7/f/fs,
  8/f/fs,
  9/2/us,
  10/u/us,
  11/2/fs,
  12/17/fs,
  13/2/us,
  14/u/us,
  15/2/us,
  16/u/us,
  17/3/fs,
  18/23/fs,
  19/f/fs,
  20/3/us,
  21/u/us,
  22/u/us,
} {
  \draw[\s] (\addr,0) rectangle node {\data} (\addr+1,0.5);
  \node[anchor=south] at (\addr+0.5,0.5) {\addr};
}
  \draw[very thick] (0,0) rectangle (2,0.5);
  \draw[very thick] (2,0) rectangle (5,0.5);
  \draw[very thick] (5,0) rectangle (9,0.5);
  \draw[very thick] (9,0) rectangle (11,0.5);
  \draw[very thick] (11,0) rectangle (13,0.5);
  \draw[very thick] (13,0) rectangle (15,0.5);
  \draw[very thick] (15,0) rectangle (17,0.5);
  \draw[very thick] (17,0) rectangle (20,0.5);
  \draw[very thick] (20,0) rectangle (23,0.5);
\foreach \i/\j in {6/11, 12/17, 18/23} {
  \path
    (\i+0.5,0)
    edge [->,thick,shorten <=2pt,shorten >=2pt,>=stealth',bend right=20]
    (\j+0.5,0);
}
\end{tikzpicture}
\end{center}
In the picture, we see an array of size~$23$,
  whose cells are indexed by $0,1,2,\ldots,22$.
The free list starts at index~$5$:
  we need to store this information somewhere, separately.
The free block starting at index~$5$ has size~$4$:
  we find the $4$ written in the cell with index~$5$.
The next free block starts at index~$11$:
  we find the $11$ written in the cell with index~$5+1$.
Similarly, at index~$11$ we have the size of a free block,
  and at index $11+1$ we have the index where the next free block starts.
And so on, until we find the index~$23$,
  which is just after the end of the array.

We will restrict the size of \emph{any} block to be~$\ge2$.
Notice that free blocks need $2$~cells,
  because their first cell contains the size of the block
  and their second cell contains the index of the next free block.
Technically, used blocks only need $1$~cell, to store their own size;
  however, allowing used blocks to be smaller than free blocks
    would make it cumbersome to transform a used block into a free one.


\subsection{First Fit, Best Fit}

When $\.{malloc}(n)$ is called,
  we begin traversing the free list to find a free block.
Let $m$ be the size of the free block.
(We will see in a moment how big $m$ should be.)
We split the free block in two:
  a used block of size $n+1$,
  and a free block of size $m-(n+1)$.
The used block starts with a cell that stores the value $n+1$,
  and continues with the $n$~cells we reserved for the user.
Of course, we need to satisfy the constraints we have on the sizes of blocks:
  $n+1\ge 2$ and $m-(n+1)\ge 2$.
We satisfy the first constraint by requesting users to call $\.{malloc}(n)$
  only with $n\ge1$.
We satisfy the second constraint by finding a free block of size $m\ge n+3$.
The \df{first fit} strategy finds the first block in the free list
  whose size is at least $n+3$.
The \df{best fit} strategy finds the smallest block in the free list
  whose size is at least $n+3$.
Of course, in both cases, a big enough free block might not exist,
  in which case \.{malloc} fails.

When $\.{free}(j)$ is called,
  we take the used block starting at index $j-1$ and insert it into the free list.
How should this insertion be done?
One option would be to simply make the new block be the first block in the free list.
This can be done quickly because it does not require traversing the free list.
Most implementations, however,
  enforce the invariant that all arrows of the free list point from left to right.
One reason is simplicity:
  With this convention we can speak of `the previous free block'
  without worrying whether we mean previous in the array or previous in the list.
To keep all arrows from left to right,
  we must traverse the free list until we find two blocks that sandwich $j-1$.
Then we can insert the freed block in the free list in its proper place.

There are two other reasons for keeping all arrows pointing right:
  (1)~it makes the traversal of the list more cache friendly, and
  (2)~it simplifies gluing free blocks.
We will speak about caches in the next lecture.
Now, let us speak about gluing blocks.

Why would we ever need to glue blocks?
Well, with the algorithm described so far,
  \.{malloc} splits blocks into smaller ones,
  and \.{free} does not change the size or the number of blocks.
Thus, after a long sequence of \.{malloc} and \.{free} operations
  we will have smaller and smaller blocks,
  making it more and more likely
    that we will not have a big enough block to satisfy the next \.{malloc}.
To counteract such external fragmentation,
  we glue free blocks that touch each-other.
In other words, we enforce the invariant that free blocks cannot touch.

\smallskip
Observe that \.{malloc} and \.{free} are rather complicated.
They are not constant time operations, as most developers assume.
You can find a C implementation for first fit, as described above,
  in the accompanying code.

\subsection{Buddy System}

In the buddy system block sizes can only be a power of two.
So, for example, if the memory has size $2^{32}$,
  then the blocks can have sizes $2^0, 2^1, 2^2, \dots, 2^{32}$.
A block of size~$2^k$ is said to have \emph{order}~$k$.
All the free blocks of order~$0$ are chained in the free list~$0$;
  all the free blocks of order~$1$ are chained in the free list~$1$;
  and so on.
Whenever a block of order~$k+1$ is split into two blocks of order~$k$,
  we need to remove it from the free list~$k+1$
  and to insert the two buddies in the free list~$k$.
Similarly, when gluing two buddies of order~$k$,
  we need to appropriately update the free lists $k$~and~$k+1$.
For each free list,
  we typically enforce the same invariants that we had for first-fit and best-fit.

\section{Garbage Collection}

With garbage collection,
  the user interface is even simpler:
  instead of two functions
    (one for allocating memory and one for deallocating memory),
  there is only one function, for allocating memory.

This topic is optional
  because few operating systems implement garbage collection.
Linux uses garbage collection internally in specific places only;
  for example, see \.{net/unix/garbage.c}.
However, this example doesn't really count
  because the garbage collector is used internally, not exposed to apps.
The Singularity operating system does implement several garbage collectors,
  \emph{and} allows apps to use them.
However, this example doesn't really count either
  because Singularity is a research prototype only.
Still,
  many high-level languages do implement garbage collection,
  and it is conceivable that this functionality
    will eventually migrate in the operating system.
So, you may want to know a bit about garbage collection.

Let us look at a couple of differences
  between garbage collection and the \.{malloc}\slash\.{free} method.

A first difference is that programmers need not worry about when to call \.{free}.
Despite what it may seem,
  such a simplification makes programming significantly easier.
The flip side is that
  the programmer doesn't get to determine when memory is deallocated.
To emphasize this difference,
  the \.{malloc}\slash\.{free} method
  is sometimes called \emph{deterministic} memory management.

%In pure functional languages there is no concept of temporal sequencing:
%  A~happens before~B\null.
%In such languages, the \.{malloc}\slash\.{free} method is unusable,
%  because \.{malloc} should happen before \.{free}.
%(Well, unusable may be a strong statement; but, certainly massively awkward.)
%So, functional languages, including non-pure ones,
%  tend to come with garbage collectors.

A second difference is that external fragmentation is less of a problem
  with garbage collection.
Most high-level languages that implement garbage collection
  distinguish between object references and memory addresses.
As a result,
  the garbage collector is allowed to change the memory address
  without changing the object reference.
In other words, the garbage collector gets to move objects around in memory,
  without the programmer ever knowing that this happens!
In the \emph{compaction} phase,
  the garbage collector moves objects so they occupy a contiguous region of memory.

Finally, programmers do need to know how their garbage collector works,
  occasionally.
For example,
  if the garbage collector interrupts the execution of the program for compaction,
  then you may have trouble implementing a real-time system,
  which must obey hard deadlines.
Other garbage collectors, however, may run concurrently,
  in a way that never interrupts the main program for long.
So, occasionally, programmers may need to explicitly choose
  which garbage collector to use.

This was quite enough about garbage collection.
If you want to know more, go and ask Richard Jones.
You are lucky to have him in Canterbury --
  he is {\bf the} garbage collection person.

\section{Exercises}

\begin{enumerate}
\item
  $\blacktriangleright$
  Run `\.{man malloc}', and read.
\item
  Look at the code example that uses \.{malloc} and \.{free}.
  Can you explain what each statement does?
\item
  Our C example reads the following input:
    `$n\ x_0\ x_1\ \ldots\ x_{n-1}\ j$',
    where $0\le j<n$.
  What if the first integer~$n$ is missing?
  Can you solve the problem in C if the input is only
    `$x_0\ x_1\ \ldots\ x_{n-1}\ j$'?
\item
  Write a C program that reverses a list of integers.
  The input is formed of $n+1$ whitespace separated integers: `$n$~$x_1$~$\ldots$~$x_n$'.
  The output is formed of $n$ space separated integers: `$x_n$~$\ldots$~$x_1$'.
  You will need to use \.{malloc} to allocate space for integers.
  Make sure you clean up after yourself.
  (Use \.{valgrind} to check you did so.)
\item
  The functions \.{malloc} and \.{free} are in fact implemented on top of \.{sbrk} and \.{mmap}.
  The low level interface is not one big array of bytes,
    but rather several big array of bytes, called \emph{arenas}.
  To get a new arena, you use \.{mmap};
  to enlarge an arena, you use \.{sbrk}.
  (Both operations may fail.)
  Take a look at \.{man~sbrk} and at \.{man mmap}.
\item
  $\blacktriangleright$
  If we want to enforce that all arrows of a free list point to the right,
    $\.{free}(j)$ must find two free blocks that sandwich $j-1$.
  What if $j-1$ comes after the last free block?
  What if $j-1$ comes before the first free block?
  Can you handle the latter case by initialising the array with some special blocks?
\item
  Implement best-fit memory allocation, in C\null.
\item
  Download the source code of \.{glibc},
    the {\bf G}NU implementation of the standard {\bf lib}rary of {\bf C}.
  Then look at the real implementation of \.{malloc},
    in \.{malloc/malloc.c}.
\item
  $\blacktriangleright$
  Which type of fragmentation tends to be a problem for the buddy system?
\item
  Implement the buddy system, in C\null.
\end{enumerate}

\section{Notes}

\citet{taocp} Volume~2, Section~2.5 covers some theory behind memory allocation.
\citet{survey-malloc} survey the state-of-the art in the 90s, making the case that theory tends to be useless in this area and one needs empirical studies.
Finally, \citet{singularity} present an operating system that does garbage collection.

% vim:spell:spelllang=en_gb:
