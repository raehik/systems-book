\chapter{Introduction to Operating Systems}\label{ch:00-introduction}

\section{Why Study Operating Systems?}

The vast majority of programmers do \emph{not} work on operating systems.
Given this, is it a good idea to invest time in learning about operating systems?
Yes.
Let's see why.

There is so much software around that no human can comprehend more than a tiny fraction of it.
How come we can keep writing software without getting hopelessly confused by its complexity?
How come the software written nowadays does so much more than the software written 50~years ago?
There is an unlikely person who'd know the answer to that last question:
  Isaac Newton.
He knew he saw far because he stood on the shoulders of giants.
He may not know what `software' is.
But, he'd probably guess that today's software stands on the shoulders of yesterday's software,
  and that's how it does more.

In fact, the whole edifice built by programmers looks like a skyscraper,
  reaching up at dizzying heights because it is carefully built layer upon layer.
Layer upon layer of abstraction.
This idea, of having \emph{layers of abstraction},
  is one of the most important ideas in computer science.
In a sense,
  the whole course is built to teach the concept of abstraction,
  by example.
Of course,
  one may wonder why would operating systems be a particularly good source of examples of abstraction.
Fair question.
To answer it, let's get back to our skyscraper.
It's silent around.
But, if you look up, and if you look really carefully (perhaps with binoculars),
  you'll see a lot of activity at the top.
That's where the builders are!
But --- hold on --- they seem to be doing something strange:
  Instead of continuing up more or less as you'd expect,
  there are several teams building in different directions.
The skyscraper has branches!
Why would they do such a thing?

It's because they have strong opinions.
Each team of builders thinks that the skyscraper should continue the way \emph{they} build it.
Each team is unable to convince the other teams that \emph{theirs} is the right way to build.
So, each team carries on with what they think is best.
Eventually, what happens over and over in history is that one or two branches outlast the others.
The others either collapse, or builders simply abandon them.
The truth is that no one knows which will survive and which will collapse.

What is this analogy supposed to mean?
The top floors of the skyscraper represent the new software.
Most active programmers work on fairly new pieces of software.
This new software is built in a variety of ways,
  and the truth is that nobody knows which of those ways is the `right' one.
Of course, everybody has their favourite way of writing software.
And, of course, everybody thinks they're right.
And, of course, some of us will be `proved' right --- it's inevitable.
Alas, that doesn't necessarily mean that some of us are better at predicting
  which new software will pass the test of time.

But, if we look into the past, we can \emph{see} which software passed the test of time.
Operating systems are at level~1 of the skyscraper.
Immense volumes of software are based on them, and they didn't crumble.
Something about them must be right.

\section{What is an Operating System?}

Many people who use computers can give examples of operating systems:
  Windows, Android, iOS, macOS, Linux, and so on.
Many people know that Android and iOS feel differently.
And many people might think that operating systems are bound to certain devices:
  Android to `Android phones',
  iOS and macOS to Apple hardware,
  Windows to Microsoft-compatible hardware,
  and so on.
But none of this addresses the question of what an operating system \emph{is}.
For an analogy,
  imagine that you asked someone what is a car,
  and you received the following answer.
Ford, Toyota, and BMW are cars.
A Ford and a BMW feel differently.
Also, Ford is manufactured in US, and BMW in Germany.
After this answer,
  you still don't know what a car \emph{is}.

\medskip

Operating systems are programs that sit between hardware and user applications.
This means that applications can afford to ask fairly high-level things,
  such as `give me $1$~MiB of memory where I can store data'.
The operating system hears such a request and takes care of all the nitty-gritty details
  necessary to make it happen on some particular hardware.

Of course, the above definition is rather vague.
In fact,
  it's vague enough that almost everybody agrees
  that it is a fair characterisation of operating systems.
We can try to be more concrete, at the risk of alienating some friends,
  who'd suddenly think we have a too narrow definition.
Nevertheless, let's try.

\df{POSIX} is a standard of what the operating system should offer applications.
It lists C functions, command line utilities, and is in general rather comprehensive.
We could define a (POSIX-compliant) operating system as being a piece of software
  that implements the POSIX interface.
But, POSIX lists for example the \.{malloc} function,
  and some people would disagree that \.{malloc} is implemented in the operating system.
Why?
Two reasons are political, another is technical.

Let's mention the political reasons only briefly.
First, Windows implements only a small subset of POSIX,
  and it is harsh to say that Windows is not an operating system.
Second, on Linux systems,
  the POSIX interface is implemented by two antagonist developer camps:
  the kernel developers and the GNU developers.
The kernel developers think the operating system is \emph{only} the kernel;
  the GNU developers (e.g., Richard Stallman) disagree.

Now back to the technical reason.
The processor can run in two modes, typically called kernel and user.
In kernel mode, the processor does what it is told.
In user mode, the processor refuses to do some things that look odd.
For example, if the code tries to access some memory it should not access,
  then the processor interrupts the execution,
  and jumps to a piece of code that handles such an `exceptional' situation.
Some people, like the kernel developers,
  prefer to define operating systems as follows:
  the operating system is the code that runs in kernel mode.

In this course we will look at some guts of the kernel,
  but also at some guts that spill out of the kernel.
In other words, we stick to the POSIX definition.

\medskip

This definition is not Windows-friendly and is not Android-friendly.
But, that's OK:
  throughout the course we shall use Linux as our primary example.
We have only one primary example to keep things simple.
Why is Linux that example?
It is the basis of Android, iOS, and macOS;
  so, if we learn about Linux we also learn about these other operating systems.
Also, Linux is open source;
  so, we can peek inside, and we can modify it.

\section{Compiling the Linux Kernel}

You do not need to have a Linux machine to look at the Linux kernel.
It is sufficient to have a Linux virtual machine.
To set up an environment that allows you
  to look at the kernel source, make modifications, and observe their effect,
  follow the steps below:
\begin{enumerate}
\item Install \href{https://www.virtualbox.org/wiki/Downloads}{VirtualBox}.
\item Create an Ubuntu virtual machine.
  For that you'll need the
    \href{https://www.ubuntu.com/download/desktop}%
      {ISO image of an Ubuntu installation DVD}.
\item Build and modify your kernel using
  \href{https://wiki.ubuntu.com/Kernel/BuildYourOwnKernel}%
    {instructions from the Ubuntu Wiki}.
\end{enumerate}
If you have any problem with the steps above,
  then ask a question in the Moodle forum.

\section{Components of an Operating System}

As mentioned above,
  processors have two modes of operation,
  called user mode and kernel mode.
Applications usually run in user mode
  but, every once in a while,
   they need to perform a task in kernel mode.
The switch between user and kernel mode
  is achieved by doing a \emph{system call}.
From the point of view of the application programmer,
  a system call is not much different from a normal function call.
In fact,
  most of the time the programmer calls a normal function
  which is part of the standard C library
  and which happens to do very little else than forwarding to a system call.
Such a function is called a \emph{wrapper} for a system call.

Beneath this interface of system calls
  lies the operating system code that does actual work.
Some of this code is architecture-dependent,
  so there is some truth to the common assumption
  that an operating system is bound to a certain type of hardware.
Let us look at an example.
\begin{center}
\begin{tabular}{@{}lrrr@{}}
\toprule
Linux~4.4 [MLOC] & \.{arch/} & \.{drivers/} & other \\
\midrule
headers (\.{*.h}) & 0.8 & 1.9 & 1.2 \\
implementation (\.{*.c}) & 1.6 & 9.9 & 4.0 \\
assembly (\.{*.S}) & 0.4 & 0.0 & 0.0 \\
\bottomrule
\end{tabular}
\end{center}
The numbers above are in MLOC:
  {\bf m}ega (millions) {\bf l}ines {\bf o}f {\bf c}ode.
We distinguish between three types of files.
C code is in files with extension \.{.h}~or~\.{.c}:
  in \.{.h} files one typically finds function signatures,
    which define the interface between different parts of the code;
  in \.{.c} files one typically finds the code that does actual work.
Assembly code is in files with extension~\.{.S}.
The kernel source is split among several directories,
  two of which are \.{arch/} and \.{drivers/}.
In \.{arch/}, one finds code that depends on the hardware;
  for example,
    one could find different implementations for a spin lock:
    one implementation for ARM, another implementation for x86.
In \.{drivers/},
  one finds code that knows how to communicate with various devices;
  for example, one could find code that knows how to interact
    with a graphic card of a certain type.

Most of the code of the Linux kernel is in the \.{drivers/} directory.
There is less code, but still a considerable amount,
  in the \.{arch/} directory,
  including assembly code.
In this module,
  we will ignore the code in the \.{arch/} and \.{drivers/} directories.
This choice is rather arbitrary:
If we had more time, we would have covered those directories as well.

\smallskip

Let us make a short digression.
Last year, I wrote about 10~thousand lines of code.
But, I spend a considerable amount of time doing things
  other than writing code.
A programmer would write more code; say, 100~thousand lines of code per year.
If such a person would be able to write the Linux kernel
  without having to re-write any code ever,
  then such a person would still need $\approx200$~years to finish.
The Linux kernel is one big piece of software,
  and it only exists because many programmers \emph{collaborated}!

\smallskip

Back from the digression.
What do we find in the `other' directories?
Mostly, code that manages the following hardware resources:
  memory, hard drive, network, processor.
\begin{center}
\begin{tabular}{@{}ll@{}}
\toprule
code to manage \dots & is in directory \dots \\
\midrule
memory & \.{mm/} \\
hard drive & \.{fs/} \\
network & \.{net/} \\
processor & \.{kernel/} \\
\bottomrule
\end{tabular}
\end{center}
Another interesting directory is \.{tools/}:
  It contains small, self-contained utilities
    that help you loot at the kernel's internals while it runs.

We will focus on ideas.
However,
  every once in a while we will see how ideas are fleshed out in code.
In the privacy of your home,
  you should spend a considerable amount of time goggling other people's code,
  if you want to become a good programmer.


\section{C versus Java}

Java was designed to look like C while avoiding some complications.
Conversely, C looks like Java, but has some extra complications.
Two important differences are memory management and undefined behaviour.
Let us discuss them in turn.
\todo{Add some details.}

\smallskip

In Java, you allocate memory using the \.{new} keyword.
In C, you allocate memory using the \.{malloc} function.
That's not much of a difference.
For deallocation, however,
  in Java you don't need to think about it but in C you do.
When you are done with using a piece of memory,
  Java will garbage collect it,
  but C will leave it to linger around until \emph{you} throw it away
    by calling the function \.{free}.

\smallskip

In Java, if you access an array out of bounds, you get an exception.
In C, the standard says that anything can happen ---
  the behaviour is undefined.
This is sometimes known as the \emph{nasal demons} problem:
  Strictly speaking,
  if you access an array out of bounds,
  a C compiler is within its right to make demons fly out of your nose.
In practice, this almost never happens.
More common behaviours are
  (1)~segmentation fault (that is, the program crashes), or
  (2)~the program continues but gives weird results.
If you want to catch such problems,
  you should use a tool called \.{valgrind}.
There are also a few other tools that achieve similar results,
  so you may want to google a bit to figure out which suits your needs best.
However, neither \.{valgrind} nor other tools are a panacea:
  there are multiple possible sources of undefined behaviour,
  and it is difficult to detect them all.

\smallskip

The best reference for learning C is still the original book~\citep{c-book}.

\section{Content}

The content is organized around four hardware components:
  memory, hard drive, network card, and processor.
For each of these,
  we will first discuss how they are seen by applications.
What applications see is not the real hardware but an idealized illusion.
We then turn to studying how the operating system creates that illusion.

The plan ahead is as follows:\todo{Add bitcoin as example app.}

\begin{itemize}
\item Lectures 1--2: introduction
  \begin{itemize}
  \item why study operating systems
  \item what is an operating system
  \item compiling the Linux kernel
  \item components of an operating system
  \item basics of the C programming language
  \end{itemize}
\item Lectures 3--5: memory
  \begin{itemize}
  \item interface offered to applications (\.{malloc}, \.{free})
  \item implementation in the operating system
    \begin{itemize}
    \item memory allocation (free list, first fit, best fit)
    \item caching
      \begin{itemize}
      \item ideal cache model
      \item memory hierarchy
      \item associativity
      \item replacement strategies: FIFO, LRU, MIN
      \end{itemize}
    \item virtual memory (page table, thrashing, TLB)
    \end{itemize}
  \end{itemize}
\item Lectures 6--7: hard drive
  \begin{itemize}
  \item interface offered to applications
    \begin{itemize}
    \item tree structure
    \item permissions
    \item high-level interface (\.{fopen}, \.{fclose}, \.{prinf}, \.{scanf})
    \item low-level interface (\.{open}, \.{close}, \.{write}, \.{read})
    \end{itemize}
  \item implementation in the operating system
    \begin{itemize}
    \item dcache, dentries, i-nodes
    \item tracking free/used space
    \item I/O scheduling
    \end{itemize}
  \end{itemize}
\item Lectures 8--9: network card
  \begin{itemize}
  \item interface offered to applications
    \begin{itemize}
    \item IP addresses, ports
    \item sockets (\.{socket}, \.{connect}, \.{bind}, \.{listen})
    \item name resolution (DNS, \.{getaddrinfo})
    \item network administration tools
      (\.{ip}, \.{tc}, \.{ss}, \.{nmap}, \.{tcpdump})
    \end{itemize}
  \item implementation in the operating system
    \begin{itemize}
    \item layers: link, network, transport, application
    \item network protocols (IP)
    \item transport protocols (TCP, UDP)
    \item routing protocols (OSPF, IS-IS, BGP) and algorithms
    \end{itemize}
  \end{itemize}
\item Lectures 10-11: processor
  \begin{itemize}
  \item interface offered to applications
    \begin{itemize}
    \item cooperative and preemptive multitasking
    \item processes (\.{fork}, \.{wait})
    \item messaging (POSIX message queues, pipes, signals)
    \item threads (\.{pthread\_create}, \.{pthread\_join})
    \item synchronization primitives
      (mutex, condition variable, monitor, semaphore, RW lock)
    \end{itemize}
  \item implementation in operating system
    \begin{itemize}
    \item schedules: delay, utilisation, context switch, thread migration, real time
    \item schedulers: round robin, CFS
    \end{itemize}
  \end{itemize}
\end{itemize}

\section{Rules}

\begin{itemize}
\item
  You should read the lecture notes.
  Sections marked$^\ast$ with an asterisk are optional.
\item
  You should do the exercises from the lecture notes.
  Important exercises are marked with $\blacktriangleright$.
\item
  On Moodle,
    there is a forum for `discussions about lecture notes, including exercises'.
  Use it for:
  \begin{itemize}
  \item asking for clarifications
  \item reporting mistakes
  \item suggesting improvements
  \item asking for help with solving the exercises
  \end{itemize}
\item
  We will have quizzes in lectures.
  These are not assessed.
\item
  Lectures are \emph{not} guaranteed to be recorded.
  Instead, MOOC-style videos are provided.
\end{itemize}

\section{Exercises}

\begin{enumerate}
\item
  $\blacktriangleright$
  Read  `\href{http://blog.regehr.org/archives/164}{Why Take an Operating Systems Course?}'
    by J Regehr.
\item Google `why study operating systems?'
\item Find a good reference for POSIX, on the Internet.
\item
  $\blacktriangleright$
  Follow the steps in `Compiling the Linux Kernel'.
\item Write a `hello world' program in C.
\item What are operating systems?
\item What is POSIX?
\item
  $\blacktriangleright$
  Get familiar with the command \.{man}, including its \.{-k} option.
\item
  $\blacktriangleright$
  Read `\href{http://blog.codinghorror.com/understanding-user-and-kernel-mode/}{Understanding User and Kernel Mode}'.
\item
  $\blacktriangleright$
  Run `\.{man syscalls}' to see which system calls are available on your computer.
  Sample a few of them, and look at their \.{man}ual pages;
  for example, run
    `\.{man chdir}',
    `\.{man kill}',
    and so on.
%\item
%  $\blacktriangleright$
%  Study the \href{http://www.ibm.com/developerworks/linux/library/l-linux-kernel/}%
%  {anatomy of the Linux kernel}.
\end{enumerate}

% vim:spell:spelllang=en_gb:
